\documentclass[utf8]{psta}% 
% Пожалуйста, уточните классификация Вашей статьи согласно УДК, ББК и MSC
\subjclass[UDC]{519.68}
\subjclass[BBC]{32.973.202-018.2}
\subjclass[2010]{97P30; 97P20, 97R40}

% Заглавие работы
\title[кратко]{Здесь должно быть заглавие}
%\title[]{} % или с коротким вариантом для колонтитула

% Фамилия, запятая, имя отчество автора 
\author{Фамилия, Имя Отчество}
% Организация, в которой выполнена статья или её часть автором
\address{Университет}
% Электронный адрес автора
\email{}
% Какими грантами поддержана работа автора
\thanks{}
%  краткая информация, в свободной форме  презентующая официальный статус автора, его научные интересы и достижения. 
\info{}
% фотография, позволяющая узнать автора в толпе участников любой конференции
%\image{}
\image{nobody}
\orcid{}    
% Аналогично для каждого из остальных авторов

% Пару строчек ключевых слов и фраз для поиска
\keywords{}
\begin{abstract}
   Здесь должна быть аннотация на русском языке
\end{abstract}

% Все метаданные должны также присутствовать на английском языке, 
% заключённые в  \selectlanguage{english}...,\selectlanguage{russian}: 

\selectlanguage{english} 
% All the same in English 
\title[Short title]{The Full Engish Title}
% Last name, coma other names 
\author{Family Name,Other Names}
% Organisation, where the work done
\address{University}
% author email
\email{}
% support notes
\thanks{}
% Other information about author only on paper language 
%\info{} %
% author photo
%\image{}
%\orcid{}
% Repeat the same fore each of other authors
%
\begin{abstract}
The abstract in English should come here.
\end{abstract}
\selectlanguage{russian} % Не забывайте отметить возврат на русский язык
% Для локального переключения на другой язык используйте команду 
% \foreignlanguage{english}{Text in English}

\begin{document}           
\maketitle   
%%%%%% Текст статьи может использовать 
\section*{Введение}
Проверяйте, пожалуйста (URL) в списке литературы \cite{PSTAmanual}!% ссылка на источник литературы
\section{}
\subsection{}
%\citеs{,,} % ссылка на несколько источников

%\begin{figure} % Never fix the place!
%\includegraphics{pic} % Без расширения, должно быть jpg, png или pdf
%%\includegraphics[width=10cm]{pic} % Если по ширине страницы
%\caption{}
%\label{}
%\end{figure}

%%%%% Нумерация библиографии в порядке цитирования, инициалы перд фамилией. 
% 
\begin{thebibliography}{30}
%
%\bibitem{Voss2006} H.~Vo\ss. ``Math mode, 2006. Extensive summary describing various mathematical constructions, both with and without the amsmath package''.
% {\scriptsize \textsf{URL:} }\url{http://mirror.ctan.org/info/math/voss/mathmode/Mathmode.pdf}
%
%% Допускается стандартный bibtex, рекомендуется форматирование AMSBIB:
%
\RBibitem{PSTAmanual} 
\by С.В. Знаменский
\preprint Руководство по стилю PSTA для авторов, подготавливающих статьи в \LaTeX
\yr 2018 
\totalpages 9
\URL http://psta.psiras.ru/read/psta-man.pdf

\end{thebibliography}
\end{document}

