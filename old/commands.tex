% Схемы для различных ассоциативных операторов. Использование примерно такое:
%
% \opscheme{\product}{\prod}{\times} -- простая схема для большинства случаев.
%
% \dopscheme{\product}{\prod}{\times}{\Dom}{\identity}{\identity} -- схема с
% декорациями. Последние три параметра украшают полученное выражение: элементы
% операции, итоговое выражение без большого оператора, итоговое выражение с
% большим оператором.
%
% \mopscheme{\product}{\prod}{\times} -- простая схема, генерирующая команду,
% которая работает с макросом, а не с названием элемента.
%
% \dmopscheme{\product}{\prod}{\times}{\identity}{\identity} -- команда с
% украшением. Украшать сам элемент не нужно (это делает макрос, передаваемый
% параметром в сгенерированную команду, поэтому украшения всего два: для
% итоговых формул.

\newcommand\opscheme[3]{%
  \NewDocumentCommand#1{mggg}{%
    \IfNoValueTF{##4}
      {\IfNoValueTF{##3}
        {\IfNoValueTF{##2}
          {\ensuremath{\nomapinset{#3}{##1}}}
          {#1{{##1} {\dots} {##2}}}}
        {#1{##1}{##2}{1}{##3}}}
       {\bigop{#2}{##1}{##2}{##3}{##4}}}}

% \dots не должны попасть под украшение для элементов (пераметр 4). Поэтому
% отдельная обработка
\newcommand\dopscheme[6]{%
  \NewDocumentCommand#1{mggg}{%
    \IfNoValueTF{##4}
      {\IfNoValueTF{##3}
        {\IfNoValueTF{##2}
          {\ensuremath{#5{\simpleinset{#3}{#4}{##1}}}}
          {\ensuremath{#5{\nomapinset{#3}{{#4{##1}} {\dots} {#4{##2}}}}}}}
        {#1{##1}{##2}{1}{##3}}}
    {\ensuremath{{#6{\bigop{#2}{#4{##1}}{##2}{##3}{##4}}}}}}}

\newcommand\mopscheme[3]{%
  \NewDocumentCommand#1{mmgg}{%
    \IfNoValueTF{##4}
      {\IfNoValueTF{##3}
        {\simpleinset{#3}{##1}{##2}}
        {#1{##1}{##2}{1}{##3}}}
    {\mbigop{#2}{##1}{##2}{##3}{##4}}}}

\newcommand\dmopscheme[5]{%
  \NewDocumentCommand#1{mmgg}{%
    \IfNoValueTF{##4}
      {\IfNoValueTF{##3}
        \ensuremath{{#4{\simpleinset{#3}{##1}{##2}}}}
        {#1{##1}{##2}{1}{##3}}}
    {\ensuremath{#5{\mbigop{#2}{##1}{##2}{##3}{##4}}}}}}

