% Определение последовательности. Штука либо показывается заданным способом 2
% аргумент, либо показывается, как аппликация функции к указанному аргументу.

\newcommand\seqscheme[2]{%
  \NewDocumentCommand#1{g}{%
    \IfNoValueTF{##1}
      {\ensuremath{#2}}
      {\ifx\relax##1\relax{\ensuremath{#2}}\else{\app{#2}{##1}}\fi}}}

\NewDocumentCommand\seqcore{mggg}{%
  \IfNoValueTF{#3}
    {\nomapinset{\ }{#1}}
    {\IfNoValueTF{#4}
      {\seqcore{#1}{#2}{1}{#4}}
      {\ensuremath{{#1}_{{#2}={#3}}^{#4}}}}}

\NewDocumentCommand\mseqcore{mmgg}{%
  \IfNoValueTF{#3}
    {\simpleinset{\ }{#1}{#2}}
    {\IfNoValueTF{#4}
      {\mseqcore{#1}{#2}{1}{#3}}
      {\ensuremath{{#1{#2}}_{{#2}={#3}}^{#4}}}}}

\newcommand\cfnscheme[2]{%
  \NewDocumentCommand#1{g}{%
    \IfNoValueTF{##1}
      {\ensuremath{#2}}
      {\ifx\relax##1\relax{\ensuremath{#2}}\else{\capp{#2}{##1}}\fi}}}

\newcommand\dcfnscheme[3]{%
  \NewDocumentCommand#1{g}{%
    \IfNoValueTF{##1}
      {\ensuremath{#2}}
      {\ifx\relax##1\relax{\ensuremath{#2}}\else{\dcapp{#2}{##1}{#3}}\fi}}}

\ifclassic
  \newcommand\fnscheme[2]{%
    \NewDocumentCommand#1{g}{%
      \IfNoValueTF{##1}
        {\ensuremath{#2}}
        {\ifx\relax##1\relax{\ensuremath{#2}}\else{\capp{#2}{##1}}\fi}}}

  \newcommand\dfnscheme[3]{%
    \NewDocumentCommand#1{g}{%
      \IfNoValueTF{##1}
        {\ensuremath{#2}}
        {\ifx\relax##1\relax{\ensuremath{#2}}\else{\dcapp{#2}{##1}{#3}}\fi}}}
\else
  \newcommand\fnscheme[2]{%
    \NewDocumentCommand#1{g}{%
      \IfNoValueTF{##1}
        {\ensuremath{#2}}
        {\ifx\relax##1\relax{\ensuremath{#2}}\else{\app{#2}{##1}}\fi}}}

  \newcommand\dfnscheme[3]{%
    \NewDocumentCommand#1{g}{%
      \IfNoValueTF{##1}
        {\ensuremath{#2}}
        {\ifx\relax##1\relax{\ensuremath{#2}}\else{\dapp{#2}{##1}{#3}}\fi}}}
\fi

