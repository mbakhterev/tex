% Схемы для различных ассоциативных операторов с «большими» обозначениями (\sum,
% \bigcup и тому подобных). Примеры использования:
%
%   \opscheme{prod}{\times}{\prod}
%   \prodx{1 2 3} → 1 \times 2 \times 3
%
%   \newcommand\pars[1]{(#1)}
%   \dopscheme{sum}{+}{\sum}{\identity}{\pars}{\identity}
%   \sumx{1 2 3} → (1 + 2 + 3)

% Расстановка индексов для больших операторов (\prod, etc).

\newcommand\bigop[5]{\ensuremath{#1_{{#3}={#4}}^{#5}{{#2}_{#3}}}}
\newcommand\bigopm[5]{\ensuremath{#1_{{#3}={#4}}^{#5}{#2{#3}}}}

% #1 -- основа для имён команд
% #2 -- символ операции
% #3 -- «большой» групповой символ операции
\newcommand\opscheme[3]{%
  \expandafter\newcommand\csname#1i\endcsname[0]{\ensuremath{#2}}

  \expandafter\newcommand\csname#1x\endcsname[1]{\nomapinset{#2}{##1}}

  \expandafter\newcommand\csname#1r\endcsname[2]{%
    \nomapinset{#2}{{##1} {\dots} {##2}}}

  \expandafter\newcommand\csname#1e\endcsname[3]{%
    \csname#1f\endcsname{##1}{##2}{1}{##3}}

  \expandafter\newcommand\csname#1f\endcsname[4]{%
    \bigop{#3}{##1}{##2}{##3}{##4}}

  \expandafter\newcommand\csname#1mx\endcsname[2]{%
    \simpleinset{#2}{##1}{##2}}

  \expandafter\newcommand\csname#1mr\endcsname[3]{%
    \nomapinset{#2}{{##1{##2}} {\dots} {##1{##3}}}}

  \expandafter\newcommand\csname#1me\endcsname[3]{%
    \csname#1mf\endcsname{##1}{##2}{1}{##3}}

  \expandafter\newcommand\csname#1mf\endcsname[4]{%
    \bigopm{#3}{##1}{##2}{##3}{##4}}
}

% #1 -- основа для имён команд
% #2 -- символ операции
% #3 -- «большой» групповой символ операции
% #4 -- поэлементная декорация
% #5 -- декорация для простых формул
% #6 -- декорация для «больших» формул
\newcommand\dopscheme[6]{%
  \expandafter\newcommand\csname#1i\endcsname[0]{\ensuremath{#2}}

  \expandafter\newcommand\csname#1x\endcsname[1]{%
    #5{\simpleinset{#2}{#4}{##1}}}

  \expandafter\newcommand\csname#1r\endcsname[2]{%
    #5{\nomapinset{#2}{{#4{##1}} {\dots} {#4{##2}}}}}

  \expandafter\newcommand\csname#1e\endcsname[3]{%
    \csname#1f\endcsname{##1}{##2}{1}{##3}}

  \expandafter\newcommand\csname#1f\endcsname[4]{%
    #6{\bigop{#3}{#4{##1}}{##2}{##3}{##4}}}

  \expandafter\newcommand\csname#1mr\endcsname[3]{%
    #5{\nomapinset{#2}{{##1{#4}{##2}} {\dots} {##1{#4}{##3}}}}}

  \expandafter\newcommand\csname#1me\endcsname[3]{%
    \csname#1mf\endcsname{##1}{##2}{1}{##3}}

  \expandafter\newcommand\csname#1mx\endcsname[2]{%
    \begingroup%
    \newcommand\mdeco[1]{##1{#4}{####1}}%
    #5{\simpleinset{#2}{\mdeco}{##2}}%
    \endgroup}

  \expandafter\newcommand\csname#1mf\endcsname[4]{%
    \begingroup%
    \newcommand\mdeco[1]{##1{#4}{####1}}%
    #6{\bigopm{#3}{\mdeco}{##2}{##3}{##4}}%
    \endgroup}
}

% Конец схем операторов.

