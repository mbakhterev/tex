% Различные простые макросы без возни с низкоуровневым TeX

\newcommand\grk[1]{\foreignlanguage{greek}{#1}}
\newcommand\eng[1]{\foreignlanguage{english}{#1}}

% \newcommand\range[2]{\overline{{#1},{#2}}}
\newcommand\range[2]{\enum{{#1} {\ldots} {#2}}}

% Шрифт для идентификаторов доменов

\newcommand\dfn[1]{\ensuremath{\mathrm{#1}}}

\newif\ifclassic
\classicfalse

\newcommand\identity[1]{#1}
\newcommand\esc[1]{#1}

\makeatletter

\def\im@end{\relax}
\def\insetmap#1{\im@loop#1 \im@end\relax}

\def\im@loop#1 #2\relax{%
  \begingroup%
  \ifx \im@end#2%
    \def\im@next{\endgroup\mapitem{#1}\relax}%
  \else%
    \def\im@next{\endgroup\mapitem{#1}\relax\expandafter\inset\im@loop#2\relax}%
  \fi%
  \im@next
}

\def\headsplitmap#1{\hs@split#1 \im@end\relax}

\def\hs@split#1 #2\relax{%
  \ifx \im@end#2%
    \maphead{#1}{\pretail\posttail}%
  \else
    \maphead{#1}{\pretail\insetmap{#2}\posttail}}

\makeatother

\newcommand\simpleinset[3]{%
  \begingroup%
  \def\inset{#1}%
  \def\mapitem{#2}%
  \ensuremath{\insetmap{#3}}%
  \endgroup}

\newcommand\nomapinset[2]{\simpleinset{#1}{\identity}{#2}}

% Произведения и кортежи

\newcommand\productone[1]{%
  \begingroup%
  \def\inset{\times}%
  \def\mapitem{\identity}%
  \ensuremath{\insetmap{#1}}%
  \endgroup}

\NewDocumentCommand\product{mggg}{%
  \IfNoValueTF{#4}
    {\productone{#1}}
    {\ensuremath{\prod_{#2=#3}^{#4}{#1_{#2}}}}}

\newcommand\mproduct[4]{\ensuremath{\textstyle{\prod}_{#2=#3}^{#4}{#1{#2}}}}

\ifclassic
  \newcommand\enum[1]{%
    \begingroup%
    \def\inset{,}%
    \def\mapitem{\identity}%
    \ensuremath{\insetmap{#1}}%
    \endgroup}
\else
  \newcommand\enum[1]{
    \begingroup%
    \def\inset{\ }%
    \def\mapitem{\identity}%
    \ensuremath{\insetmap{#1}}%
    \endgroup}
\fi

\newcommand\tuple[1]{\ensuremath{[\enum{#1}]}}
\newcommand\Tuple[1]{\ensuremath{\left[\enum{#1}\right]}}

\newcommand\prj[2]{\ensuremath{\pi^{#1}_{\enum{#2}}}}

% Суммы

\newcommand\oneofone[1]{%
  \begingroup%
  \def\inset{+}%
  \def\mapitem{\identity}%
  \ensuremath{\insetmap{#1}}%
  \endgroup}

\NewDocumentCommand\oneof{mggg}{%
  \IfNoValueTF{#4}
    {\oneofone{#1}}
    {\ensuremath{\sum_{#2=#3}^{#4}{#1_{#2}}}}}

\newcommand\inj[2]{\ensuremath{\epsilon^{#1}_{\enum{#2}}}}

\NewDocumentCommand\tailinj{mmg}{%
  \IfNoValueTF{#3}
    {\ensuremath{#1\rightarrowtail#2}}
    {\ensuremath{\tailinj{#1}{#2}\ {#3}}}}

% Λ-абстракции

\newcommand\fndom[1]{%
  \begingroup%
  \def\inset{\to}%
  \def\mapitem{\identity}%
  \ensuremath{\insetmap{#1}}%
  \endgroup}

\newcommand\lam[1]{\ensuremath{\lambda#1\ldotp}}

\newcommand{\capp}[1]{%
  \begingroup%
  \def\inset{,}%
  \def\mapitem{\identity}%
  \def\maphead{\identity}%
  \def\pretail{(}%
  \def\posttail{)}%
  \ensuremath{\headsplitmap{#1}}%
  \endgroup}

\ifclassic
  \newcommand\app[1]{\capp{#1}}
  \newcommand\papp[1]{\capp{#1}}
\else
  \newcommand{\app}[1]{%
    \begingroup%
    \def\inset{\,}%
    \def\mapitem{\identity}%
    \ensuremath{\insetmap{#1}}%
    \endgroup}

  \newcommand{\papp}[1]{\ensuremath{(\app{#1})}}
\fi

% S-выражения

\makeatletter

\def\get@first#1#2\relax{#1}

\newcommand\codeitem[1]{%
  \expandafter\ifx\get@first#1\relax\esc
    {#1}
  \else
    {\expandafter\ifx\get@first#1\relax\sexp
       {#1}
     \else
       {\expandafter\ifx\get@first#1\relax\satom
          {#1}
        \else
          \text{\texttt{#1}}\fi}\fi}\fi}


\makeatother

\newcommand\baresexp[1]{%
  \begingroup%
  \def\inset{\text{\texttt{\ }}}%
  \def\mapitem{\codeitem}%
  \ensuremath{\insetmap{#1}}%
  \endgroup}

\newcommand\sexp[1]{\ensuremath{\text{\texttt{(}}\baresexp{#1}\text{\texttt{)}}}}

\newcommand\satom[1]{\ensuremath{\text{\texttt{#1}}}}

\newcommand\fn[1]{\app{#1}}

% \newcommand\fappseq[1]{\intake{\code}{\codespace}{#1}}
% \newcommand\fabsseq[1]{\intake{\mbox{\texttt{\textgreek{λ}}}\code}{\codespace}{#1}\codespace}
% 
% \newcommand{\fabs}[2]{\code{(}{\fabsseq{#1}\code{#2}}\code{)}}
% \newcommand{\fapp}[1]{\code{(}{\fappseq{#1}}\code{)}}

\newcommand\clam[0]{\ensuremath{\text{\texttt{λ}}}}
\newcommand\fapp[1]{\sexp{#1}}
\newcommand\fabs[2]{\sexp{\clam{} {#1} #2}}

% CSP

\newcommand\procop[0]{\ensuremath{\twoheadrightarrow}}

\newcommand\proc[1]{%
  \begingroup%
  \def\inset{\procop}%
  \def\mapitem{\identity}%
  \ensuremath{\insetmap{#1}}%
  \endgroup}

\newcommand\pc[1]{\proc{#1}}

\newcommand\pevtone[1]{\ensuremath{\text{\texttt{#1}}}}

\NewDocumentCommand\pevt{mg}{%
  \IfNoValueTF{#2}
    {\pevtone{#1}}
    {\ensuremath{{\pevtone{#1}{:}{#2}}}}}

\newcommand\pparop{\ensuremath{\parallel}}
\newcommand\ppar[1]{%
  \begingroup%
  \def\inset{\pparop}%
  \def\mapitem{\identity}%
  \ensuremath{\insetmap{#1}}%
  \endgroup}

\newcommand{\pord}[1]{\inscribe{\sqsubseteq}{#1}}
\newcommand{\psel}[1]{\inscribe{\mid}{#1}}
\newcommand{\pmurec}[1]{\mu{#1}\ldotp\ }

\newcommand{\tr}[1]{\seq{#1}}

% \newcommand{\palpha}[1]{{\mbox{\texttt{α}}{#1}}}
% \newcommand{\tproc}[1]{{\mbox{\texttt{ι}}{#1}}}

\newcommand{\palpha}[1]{\text{\textgreek{α}}#1}
\newcommand{\tproc}[1]{\text{\textgreek{ι}}#1}

\newcommand{\tcat}[1]{\sqcat{#1}}

\newcommand{\tord}[1]{\inscribe{\le}{#1}}

\newcommand{\PC}[1]{\tuple{{\palpha{#1}} {\tproc{#1}}}}

\newcommand\sqcat[1]{%
  \begingroup%
  \def\inset{\cdot}%
  \def\mapitem{\identity}%
  \ensuremath{\insetmap{#1}}%
  \endgroup}

\newcommand\seq[1]{\tuple{#1}}

% \newcommand\term[1]{\fn{TERM {\unite{#1}}}}

\newcommand\term[1]{\fn{TERM #1}}
% \newcommand\tval[1]{\fn{val #1}}

\makeatletter

\def\extract@code#1#2\relax{#1}

\newcommand\code[1]{\relax%
  \expandafter\ifx\extract@code#1\relax\identity%
    {#1}%
  \else%
    \mbox{\texttt{#1}}%
  \fi\relax}

\makeatother

\newcommand\bbind[2]{\ensuremath{{#1}/{#2}}}
\newcommand\bind[2]{\ensuremath{[\bbind{#1}{#2}]}}
\newcommand\bindc[2]{\tuple{\bind{\code{#1}}{\code{#2}}}}

\newcommand\codespace[0]{\mbox{\texttt{ }}}

\newcommand{\bred}[1]{\inscribe{\mathrel{{\to}_\beta}}{#1}}
\newcommand{\aeq}[1]{\inscribe{\mathrel{{=}_\alpha}}{#1}}

% \newcommand{\tuple}[1]{\seq{#1}}
% \newcommand\tlen[1]{\card{\dom{#1}}}
% \newcommand\tlen[1]{{#1_{\#}}}
\newcommand\tend[1]{{#1}_{\tlen{#1}-1}}

\newcommand\tlen[1]{\capp{\mathrm{len} {#1}}}

%  \begingroup\def\inset{\times} \def\doword{} {\words#1}\endgroup}

\newcommand{\rel}[2]{\inscribe{\mathrel{#1}}{#2}}
\newcommand{\relation}[2]{\rel{#1}{#2}}

\newcommand{\card}[1]{\left|#1\right|}

\newcommand{\forevery}[1]{\forall{#1}\ldotp\ }
\newcommand{\thereis}[1]{\exists{#1}\ldotp\ }
% \newcommand{\lam}[1]{\text{\textgreek{λ}}#1\ldotp}

\newcommand{\coll}[2]{\left\{{{#1} \mid {#2}}\right\}}

\newcommand\nats{\mathbb{N}}

\newcommand{\bottom}[0]{{\bot}}
\newcommand{\dless}[0]{\sqsubseteq}
\newcommand{\dord}[1]{\inscribe{\dless}{#1}}
\newcommand{\dbot}[1]{{\bottom_{#1}}}

% \newcommand\all[1]{\inscribe{\wedge}{#1}}
% \newcommand\any[1]{\inscribe{\vee}{#1}}

\newcommand\all[1]{%
  \begingroup%
  \def\inset{\wedge}%
  \def\mapitem{\identity}%
  \ensuremath{\insetmap{#1}}%
  \endgroup}

\newcommand\any[1]{%
  \begingroup%
  \def\inset{\vee}%
  \def\mapitem{\identity}%
  \ensuremath{\insetmap{#1}}%
  \endgroup}

\newcommand\allop{\all{{} {}}}
\newcommand\anyop{\any{{} {}}}

% \newcommand{\join}[1]{\inscribe{\cap}{#1}}
% \newcommand{\unite}[1]{\inscribe{\cup}{#1}}

\newcommand\join[1]{%
  \begingroup%
  \def\inset{\cap}%
  \def\mapitem{\identity}%
  \ensuremath{\insetmap{#1}}%
  \endgroup}

\newcommand\unite[1]{%
  \begingroup%
  \def\inset{\cup}%
  \def\mapitem{\identity}%
  \ensuremath{\insetmap{#1}}%
  \endgroup}

\newcommand{\lub}[1]{\textstyle\bigsqcup{#1}}

\ifclassic
  \newcommand{\cpars}[1]{\left(#1\right)}
  \newcommand{\muset}[1]{\mu\left\{{#1}\right\}}
\else
  \newcommand{\cpars}[1]{#1}
  \newcommand{\muset}[1]{\mu{#1}}
\fi

\newcommand\fix[1]{{\mbox{fix}_{#1}\,}}
\newcommand\dom[1]{{\mbox{dom}\,{#1}}}


\newcommand\Nats{\nats}
% \newcommand\Nless[2]{{\all{{{#2} \in \Nats} {{#1} \le {#2}}}}}
% \newcommand\Nless[1]{{\nats_{{} < {#1}}}}

\newcommand\cpnmode[1]{\fn{mode #1}}
\newcommand\cpntype[1]{\fn{Type #1}}
\newcommand\cpnready[1]{\fn{ready #1}}

\newcommand\holesign{ 
  \mathchoice
    {\mbox{§}}
    {\mbox{§}}
    {\mbox{\scriptsize §}}
    {\mbox{\tiny §}}}

\newcommand\hole[1]{\holesign{#1}}

\newcommand\tlanext[1]{{{#1}'}}

\newcommand\tvar[1]{\fn{VAR #1}}
\newcommand\tlact[1]{\mathcal{#1}}
\newcommand\tlast[2]{\left[#1\right]_{#2}}
\newcommand\tlalways[1]{\square{#1}}

\newcommand\St{\mathbf{St}}

\newcommand\bydef{\overset{\mbox{\tiny def}}{=}}

\newcommand\iseq[3]{\tuple{#1}_{{#2}\in {#3}}}
\newcommand\nseq[2]{\iseq{#1}{#2}{\nats}}
\newcommand\bseq[4]{\iseq{\bind{#1_{#3}}{#2_{#3}}}{#3}{\Nless{#4}}}

\newcommand\R{\mathbf{R}}

\newcommand\tnothing{}
\newcommand\tsubst[1]{\left[#1\right]}
\newcommand\tval[1]{\splitinscribe\identity\identity\tsubst\tnothing#1\relax}

\newcommand\tlaval[1]{\splitinscribe\tlact\identity\tsubst\tnothing#1\relax}

\newcommand\upto[1]{\downarrow{#1}}
% \newcommand\uptoex[1]{\upto{({#1}-1)}}
% \newcommand\uptoex[1]{\text{\v{$\downarrow$}}{#1}}
\newcommand\uptoex[1]{{\hat{{\downarrow}}}{#1}}
% \newcommand\Nless[1]{\set{{\cdot}_{\Nats} < {#1}}}
\newcommand\Nless[1]{\uptoex{#1}}

\newcommand\stx[1]{\ensuremath{\text{#1}}}
\newcommand\bstx[1]{\ensuremath{{\text{#1}}_{\text{B}}}}

\newcommand\tcode[1]{\lstinline`#1`}
\newcommand\mcode[1]{\mbox{\tcode{#1}}}
\newcommand\mcsp{\mbox{\texttt{ }}}

\newcommand\codepair[2]{\(\frac{\mathrm{#1}}{\mathrm{#2}}\)}
% \newcommand\codepair[2]{\({\mathrm{#1}}_{\mathrm{#2}}\)}
% \newcommand\codepair[2]{{#1}/{#2}}

\newcommand\mpi[1]{\codepair{MPI}{#1}}
\newcommand\ride[1]{\codepair{RiDE}{#1}}
\newcommand\hadoop[1]{\codepair{Hadoop}{#1}}

\newcommand\uid[1]{\fn{{\text{\texttt{uid}}} {#1}}}
\newcommand\UID[1]{\fn{{\text{\texttt{UID}}} {#1}}}
\newcommand\node[1]{\fn{{\text{\texttt{node}}} {#1}}}

\newcommand\eval[1]{\llbracket {#1} \rrbracket}

\newenvironment{centercode}%
  {\begin{center}\begin{tabular}{c}}%
  {\end{tabular}\end{center}}

\newcommand\wa[1]{\ensuremath{{#1}_w}}

\newcommand\Input[1]{\fn{I {#1}}}
\newcommand\Wrte[1]{\fn{W {#1}}}
\newcommand\Img[1]{\text{Im}\,{#1}}

\providecommand\set[1]{\ensuremath{\{#1\}}}

\newcommand\In[1]{#1}
\newcommand\Wr[1]{#1}

\newcommand\lx[1]{\grk{l}-#1}
\newcommand\px[1]{\grk{p}-#1}

\newcommand\im[1]{\capp{\mathrm{im} {#1}}}
