% \usepackage{iftex}
% \usepackage{amsmath}
\usepackage{xparse}

% Различные простые макросы без возни с низкоуровневым TeX

\newcommand\grk[1]{\foreignlanguage{greek}{#1}}
\newcommand\eng[1]{\foreignlanguage{english}{#1}}

\newcommand\range[2]{\enumr{#1}{#2}}

\newif\ifclassic
\classicfalse

\newcommand\identity[1]{#1}
\newcommand\esc[1]{#1}

\input{engine}

% СХЕМЫ. Схемами названы генераторы новых команд по некоторым схемам. Очень
% много схожего между, например, \product и \unite: ассоциативные операторы,
% цепочки применений которых отличаются только лишь знаком, между элементами.
% Поэтому такие определения выделены в схемы.
%
% В предыдущем варианте схемы генерировали команды с переменным количеством
% аргументов при помощи \NewDocumentCommand. Это экономит идетификаторы команды,
% но довольно часто требует завершать обращения к командам \relax-ами. Что
% представляется неудобным. Кроме того, ощутимо (на слабых процессорах)
% увеличивается время компиляции. Поэтому текущая версия схем предлагает
% генерацию простых команд при помощи \newcommand. Формируемые команды
% различаются суффиксами.
%
%   i -- некоторый Identifier. Команда без аргументов, выдающая обозначение
%   операции. Для функций, например, это имя функции. 
%
%   x -- простое eXpression. Команда с одним аргументом (который может быть
%   списком, разделённых пробелом элементов, формирующая простое
%   выражение. \sumx{1 2 3} → 1 + 2 + 3.
%
%   r -- Range-выражения. Команда с двумя параметрами, которая должна изобразить
%   запись с троеточкой. \sumr{1}{10} = 1 + ... + 10.
%
%   e -- выражение с Enumeration. Очень часто индексы перечисления должны
%   пробегать от 1 до n, поэтому у команды 3 параметра: основа, имя индекса и
%   имя верхней границы. \sume{x}{i}{n} → \sum_{i=1}^{n}{x_i}.
%
%   f -- выражение From. Тоже перечисление, но с указанным нижним индексом.
%   Параметра, поэтому 4.
%
% Дополнительный суффикс m (mx, me, mf) означает команду, которая, первым
% аргументом берёт некий макрос, в который подставляется индекс или
% перечисляемые аргументы (будем говорить, что макрос применяется поэлементно):
%
%   \newcommand\spi[1]{{sp}_{#1}}\summx{a b c} → sp_a + sp_b + sp_c
%
% Необходимости в большем разнообразии суффиксов пока нет.
%
% Сами схемы могут быть в двух вариантах: простые и с декорациями (префикс d в
% названии схемы). Схемы с декорациями принимают дополнительные параметры:
% поэлементная декорация, декорация для простых выражения и для «больших»
% выражений (для \sum, например, нужны большие скобки). Декорации -- это команды
% с одним аргументом.
%
% Версии m команд с декорациями ожидают макроса для поэлементной обработки,
% который принимает два аргумента: декорацию (первый) и элемент (второй
% аргумент).
%
% Больше о схемах сказать нечего.

\input{schemes/bigops}
% 2. Схема с индексами. На случай, когда лень писать полное определение.

\newcommand\idxscheme[2]{\newcommand#1[1]{\ensuremath{{#2}_{##1}}}}

% Нечто вроде
%
%   \newcommand\midxscheme[2]{\newcommand#1[1]{\ensuremath{#2{##1}}}}
%
% здесь бессмысленно.

% Конец схемы с индексами

\input{schemes/keywords}
% % Схемы для определения функций. Для классического и аппликативного синтаксиса. 

\newcommand\indexes[4]{{#1}_{{#2}={#3}}^{#4}}
\newcommand\indexesm[4]{{#1{#2}}_{{#2}={#3}}^{#4}}

\newcommand\fnscheme[2]{%
  \expandafter\newcommand\csname#1i\endcsname[0]{%
    \ensuremath{#2}}

  \expandafter\newcommand\csname#1x\endcsname[1]{%
    \Pars{{#2}\ \nomapinset{\ }{##1}}}

  \expandafter\newcommand\csname#1r\endcsname[2]{%
    \Pars{\nomapinset{\ }{{#2} {##1} {\dots} {##2}}}}

  \expandafter\newcommand\csname#1e\endcsname[3]{%
    \csname#1f\endcsname{##1}{##2}{1}{##3}}

  \expandafter\newcommand\csname#1f\endcsname[4]{%
    \Pars{\nomapinset{\ }{{#2} {\indexes{##1}{##2}{##3}{##4}}}}}

  \expandafter\newcommand\csname#1mx\endcsname[2]{%
    \Pars{{#2}\ \simpleinset{\ }{##1}{##2}}}

  \expandafter\newcommand\csname#1mr\endcsname[3]{%
    \Pars{\nomapinset{\ }{{#2} {##1{##2}} {\dots} {##1{##3}}}}}

  \expandafter\newcommand\csname#1me\endcsname[3]{%
    \csname#1mf\endcsname{##1}{##2}{1}{##3}}

  \expandafter\newcommand\csname#1mf\endcsname[4]{%
    \Pars{\nomapinset{\ }{{#2} {\indexesm{##1}{##2}{##3}{##4}}}}}
}

\newcommand\cfnscheme[2]{%
  \expandafter\newcommand\csname#1i\endcsname[0]{%
    \ensuremath{#2}}

  \expandafter\newcommand\csname#1x\endcsname[1]{%
    \ensuremath{{#2}\!\Pars{\nomapinset{,}{##1}}}}

  \expandafter\newcommand\csname#1r\endcsname[2]{%
    \ensuremath{{#2}\!\Pars{\nomapinset{,}{{##1} {\dots} {##2}}}}}

  \expandafter\newcommand\csname#1e\endcsname[3]{%
    \csname#1f\endcsname{##1}{##2}{1}{##3}}

  \expandafter\newcommand\csname#1f\endcsname[4]{%
    \ensuremath{{#2}\!\Pars{\indexes{##1}{##2}{##3}{##4}}}}

  \expandafter\newcommand\csname#1mx\endcsname[2]{%
    \ensuremath{{#2}\!\Pars{\simpleinset{,}{##1}{##2}}}}

  \expandafter\newcommand\csname#1mr\endcsname[3]{%
    \ensuremath{{#2}\!\Pars{\nomapinset{,}{{##1{##2}} {\dots} {##1{##3}}}}}}

  \expandafter\newcommand\csname#1me\endcsname[3]{%
    \csname#1mf\endcsname{##1}{##2}{1}{##3}}

  \expandafter\newcommand\csname#1mf\endcsname[4]{%
    \ensuremath{{#2}\!\Pars{\indexesm{##1}{##2}{##3}{##4}}}}
}


\input{commands/sets}
% Домены

\makeatletter

\newcommand\domitem[1]{%
  \ifx\relax#1\relax
    {}
  \else\expandafter\ifx\get@first#1\relax\esc
    {#1}
  \else\expandafter\ifx\get@first#1\relax\Dom
    {#1}
  \else\expandafter\ifx\get@first#1\relax\DNm
    {#1}
  \else\expandafter\ifx\get@first#1\relax\KStar
    {#1}
  \else\expandafter\ifx\get@first#1\relax\KPlus
    {#1}
  \else\expandafter\ifx\get@first#1\relax\DProd
    {#1}
  \else\expandafter\ifx\get@first#1\relax\DSum
    {#1}
  \else\expandafter\ifx\get@first#1\relax\DSet
    {#1}
  \else
    \mathrm{#1}
  \fi\fi\fi\fi\fi\fi\fi\fi\fi}

\makeatother

% Стиль имён доменов

\newcommand\DNm[1]{\ensuremath{\domitem{#1}}}

\newcommand\Dom[1]{\ensuremath{{#1}_\bot}}

\newcommand\DBot[1]{\ensuremath{\bot_{\DNm{#1}}}}

\newcommand\DABot[1]{\ensuremath{\bot_{\atom{#1}}}}

% Скобки для доменных выражений

\newcommand\dset[1]{\Dom{\set{#1}}}
\newcommand\DSet[1]{\Dom{\Bras{#1}}}

\newcommand\dpars[1]{\Dom{\pars{#1}}}
\newcommand\DPars[1]{\Dom{\Pars{#1}}}

% Звёздочка Клини для переменных и доменов

\newcommand\kstar[1]{\ensuremath{{#1}^*}}
\newcommand\kplus[1]{\ensuremath{{#1}^+}}
\newcommand\KStar[1]{\kstar{\Dom{#1}}}
\newcommand\KPlus[1]{\kplus{\Dom{#1}}}

\dopscheme{DProd}{\times}{\prod}{\DNm}{\identity}{\identity}
\dopscheme{DSum}{+}{\sum}{\DNm}{\dpars}{\DPars}

% Произведения доменов

% \opscheme{prod}{\times}{\prod}

\newcommand\prj[2]{\ensuremath{\pi^{#1}_{\enumx{#2}}}}

% Суммы доменов

\dopscheme{oneof}{\sum}{+}{\identity}{\pars}{\Pars}

\newcommand\inj[2]{\ensuremath{\epsilon^{#1}_{\enumx{#2}}}}

% Различные скобки

\newcommand\pars[1]{\ensuremath{(#1)}}
\newcommand\Pars[1]{\ensuremath{\left(#1\right)}}

\newcommand\brkt[1]{\ensuremath{[#1]}}
\newcommand\Brkt[1]{\ensuremath{\left[#1\right]}}

\newcommand\bras[1]{\ensuremath{\{#1\}}}
\newcommand\Bras[1]{\ensuremath{\left\{#1\right\}}}

\newcommand\angs[1]{\ensuremath{\langle{#1}\rangle}}
\newcommand\Angs[1]{\ensuremath{\left\langle{#1}\right\rangle}}

\newcommand\spars[1]{\ensuremath{\atom{(}{#1}\atom{)}}}

% Схемы для определения функций. Для классического и аппликативного синтаксиса. 

\newcommand\indexes[4]{{#1}_{{#2}={#3}}^{#4}}
\newcommand\indexesm[4]{{#1{#2}}_{{#2}={#3}}^{#4}}

\newcommand\fnscheme[2]{%
  \expandafter\newcommand\csname#1i\endcsname[0]{%
    \ensuremath{#2}}

  \expandafter\newcommand\csname#1x\endcsname[1]{%
    \Pars{{#2}\ \nomapinset{\ }{##1}}}

  \expandafter\newcommand\csname#1r\endcsname[2]{%
    \Pars{\nomapinset{\ }{{#2} {##1} {\dots} {##2}}}}

  \expandafter\newcommand\csname#1e\endcsname[3]{%
    \csname#1f\endcsname{##1}{##2}{1}{##3}}

  \expandafter\newcommand\csname#1f\endcsname[4]{%
    \Pars{\nomapinset{\ }{{#2} {\indexes{##1}{##2}{##3}{##4}}}}}

  \expandafter\newcommand\csname#1mx\endcsname[2]{%
    \Pars{{#2}\ \simpleinset{\ }{##1}{##2}}}

  \expandafter\newcommand\csname#1mr\endcsname[3]{%
    \Pars{\nomapinset{\ }{{#2} {##1{##2}} {\dots} {##1{##3}}}}}

  \expandafter\newcommand\csname#1me\endcsname[3]{%
    \csname#1mf\endcsname{##1}{##2}{1}{##3}}

  \expandafter\newcommand\csname#1mf\endcsname[4]{%
    \Pars{\nomapinset{\ }{{#2} {\indexesm{##1}{##2}{##3}{##4}}}}}
}

\newcommand\cfnscheme[2]{%
  \expandafter\newcommand\csname#1i\endcsname[0]{%
    \ensuremath{#2}}

  \expandafter\newcommand\csname#1x\endcsname[1]{%
    \ensuremath{{#2}\!\Pars{\nomapinset{,}{##1}}}}

  \expandafter\newcommand\csname#1r\endcsname[2]{%
    \ensuremath{{#2}\!\Pars{\nomapinset{,}{{##1} {\dots} {##2}}}}}

  \expandafter\newcommand\csname#1e\endcsname[3]{%
    \csname#1f\endcsname{##1}{##2}{1}{##3}}

  \expandafter\newcommand\csname#1f\endcsname[4]{%
    \ensuremath{{#2}\!\Pars{\indexes{##1}{##2}{##3}{##4}}}}

  \expandafter\newcommand\csname#1mx\endcsname[2]{%
    \ensuremath{{#2}\!\Pars{\simpleinset{,}{##1}{##2}}}}

  \expandafter\newcommand\csname#1mr\endcsname[3]{%
    \ensuremath{{#2}\!\Pars{\nomapinset{,}{{##1{##2}} {\dots} {##1{##3}}}}}}

  \expandafter\newcommand\csname#1me\endcsname[3]{%
    \csname#1mf\endcsname{##1}{##2}{1}{##3}}

  \expandafter\newcommand\csname#1mf\endcsname[4]{%
    \ensuremath{{#2}\!\Pars{\indexesm{##1}{##2}{##3}{##4}}}}
}

% S-выражения. Команды и схемы. Схемы устроены следующим образом:
% \sexpscheme{keyword}. После чего строится набор команд со стандартными
% суффиксами i, [m]x, [m]e, [m]f. Эти команды рисуют s-выражение, в котором
% элементы записаны математически. Например,
%
%   \sexpscheme{run}\runx{a b} → \spars{\atom{run}\codespace a\codespace b}
%
% где, codespace и spars машинописные пробел и скобки. Кроме этой основной
% формы, определяется команда с суффиксом cx, в котором все элементы будут
% декорированы codeitem, и команды с суффиксами [m]bx, [m]be, [m]bf, в
% котором s-выражения не взяты в скобки spars. Это может быть полезно, как
% показывает документация Guile.
%
% В сложных декорированных схемах dsexpscheme пока не видно никакого смысла,
% потому что в s-выражениях и без того много оформления.
%
% Кроме схем нужны соответствующие конструкции для формирования произвольных
% s-выражений. Суффикс x в наборе этих команд не упомянут.

\newcommand\sexpscheme[1]{%
  \expandafter\newcommand\csname#1bxc\endcsname[1]
    {\simpleinset{\codespace}{\codeitem}{{#1} ##1}}

  \expandafter\newcommand\csname#1xc\endcsname[1]
    {\spars{\csname#1bxc\endcsname{##1}}}


  \expandafter\newcommand\csname#1bx\endcsname[1]
    {\nomapinset{\codespace}{\codeitem{#1} ##1}}

  \expandafter\newcommand\csname#1mbx\endcsname[2]
    {\ensuremath{\codeitem{#1}\codepace{\simpleinset{\codespace}{##1}{##2}}}}

  \expandafter\newcommand\csname#1x\endcsname[1]
    {\spars{\csname#1bx\endcsname{##1}}}

  \expandafter\newcommand\csname#1mx\endcsname[2]
    {\spars{\csname#1mbx\endcsname{##1}{##2}}}


  \expandafter\newcommand\csname#1br\endcsname[2]
    {\nomapinset{\codespace}{\codeitem{#1} {##1} {\ldots} {##2}}}

  \expandafter\newcommand\csname#1mbr\endcsname[3]
    {\csname#1br\endcsname{##1{##2}}{##1{##3}}}

  \expandafter\newcommand\csname#1r\endcsname[2]
    {\spars{\csname#1br\endcsname{##1}{##2}}}

  \expandafter\newcommand\csname#1mr\endcsname[3]
    {\spars{\csname#1mbr\endcsname{##1}{##2}{##3}}}


  \expandafter\newcommand\csname#1be\endcsname[3]
    {\csname#1bf\endcsname{##1}{##2}{1}{##3}}

  \expandafter\newcommand\csname#1mbe\endcsname[3]
    {\csname#1mbf\endcsname{##1}{##2}{1}{##3}}

  \expandafter\newcommand\csname#1e\endcsname[3]
    {\csname#1f\endcsname{##1}{##2}{1}{##3}}

  \expandafter\newcommand\csname#1me\endcsname[3]
    {\csname#1mf\endcsname{##1}{##2}{1}{##3}}


  \expandafter\newcommand\csname#1bf\endcsname[4]
    {\codeitem{#1}\codespace\sexpbf{##1}{##2}{##3}{##4}}

  \expandafter\newcommand\csname#1mbf\endcsname[4]
    {\codeitem{#1}\codespcae\sexpmbf{##1}{##2}{##3}{##4}}

  \expandafter\newcommand\csname#1f\endcsname[4]
    {\spars{\csname#1bf\endcsname{##1}{##2}{##3}{##4}}}

  \expandafter\newcommand\csname#1mf\endcsname[4]
    {\spars{\csname#1mbf\endcsname{##1}{##2}{##3}{##4}}}
}

\def\codespace{\text{\texttt{ }}}

\newcommand\sexpbc[1]{\simpleinset{\codespace}{\codeitem}{#1}}
\newcommand\sexpc[1]{\spars{\sexpbc{#1}}}

\newcommand\sexpb[1]{\nomapinset{\codespace}{#1}}
\newcommand\sexpmb[2]{\simpleinset{\codespace}{#1}{#2}}
\newcommand\sexp[1]{\spars{\sexpb{#1}}}
\newcommand\sexpm[2]{\spars{\sexpmb{#1}{#2}}}

\newcommand\sexpbr[2]{\nomapinset{\codespace}{{#1} {\ldots} {#2}}}
\newcommand\sexpmbr[3]{\nomapinset{\codespace}{{#1{#2}} {\ldots} {#1{#3}}}}
\newcommand\sexpr[2]{\spars{\sexpbr{#1}{#2}}}
\newcommand\sexpmr[3]{\spars{\sexpmbr{#1}{#2}{#3}}}

\newcommand\sexpbe[3]{\sexpbf{#1}{#2}{1}{#3}}
\newcommand\sexpmbe[4]{\sexpmbe{#1}{#2}{#3}{#4}}
\newcommand\sexpe[3]{\sexpf{#1}{#2}{1}{#3}}
\newcommand\sexpme[4]{\sexpmf{#1}{#2}{#3}{#4}}

\newcommand\sexpbf[4]{\ensuremath{{#1}_{{#2} = {#3}}^{#4}}}
\newcommand\sexpmbf[4]{\ensuremath{{#1{#2}}_{{#2} = {#3}}^{#4}}}
\newcommand\sexpf[4]{\spars{\sexpbf{#1}{#2}{#3}{#4}}}
\newcommand\sexpmf[4]{\spars{\sexpmbf{#1}{#2}{#3}{#4}}}

\makeatletter

\newcommand\codeitem[1]{%
  \ifx\relax#1\relax
    \text{\texttt{ }}
  \else\expandafter\ifx\get@first#1\relax\esc
    {#1}
  \else\expandafter\ifx\get@first#1\relax\sexp
    {#1}
  \else\expandafter\ifx\get@first#1\relax\satom
    {#1}
  \else\expandafter\ifx\get@first#1\relax\atom
    {#1}
  \else
    \text{\texttt{#1}}
  \fi\fi\fi\fi\fi}

\makeatother

\newcommand\baresexp[1]{%
  \begingroup%
  \def\inset{\text{\texttt{\ }}}%
  \def\mapitem{\codeitem}%
  \ensuremath{\insetmap{#1}}%
  \endgroup}

% \newcommand\satom[1]{\ensuremath{\text{\texttt{#1}}}}
% \newcommand\sexp[1]{\ensuremath{\text{\texttt{(}}\baresexp{#1}\text{\texttt{)}}}}

% \newcommand\sexp[1]{\spars{\baresexp{#1}}}

\newcommand\atom[1]{\ensuremath{\text{\texttt{#1}}}}

\newcommand\lexp[2]{\sexp{\grk{l} #1 #2}}

\newcommand\fn[1]{\app{#1}}

\newcommand\clam[0]{\ensuremath{\text{\texttt{λ}}}}
\newcommand\fapp[1]{\sexp{#1}}
\newcommand\fabs[2]{\sexp{\clam{} {#1} #2}}


\NewDocumentCommand\tailinj{mmg}{%
  \IfNoValueTF{#3}
    {\ensuremath{#1\rightarrowtail#2}}
    {\ensuremath{\tailinj{#1}{#2}\ {#3}}}}

% CSP

% \newcommand\procop[0]{\ensuremath{\twoheadrightarrow}}

% \newcommand\proc[1]{%
%   \begingroup%
%   \def\inset{\procop}%
%   \def\mapitem{\identity}%
%   \ensuremath{\insetmap{#1}}%
%   \endgroup}

\newcommand\pc[1]{\proc{#1}}

\newcommand\pevtone[1]{\ensuremath{\text{\texttt{#1}}}}

\NewDocumentCommand\pevt{mg}{%
  \IfNoValueTF{#2}
    {\pevtone{#1}}
    {\ensuremath{{\pevtone{#1}{:}{#2}}}}}

\newcommand\pparop{\ensuremath{\parallel}}
\newcommand\ppar[1]{%
  \begingroup%
  \def\inset{\pparop}%
  \def\mapitem{\identity}%
  \ensuremath{\insetmap{#1}}%
  \endgroup}

\newcommand{\pord}[1]{\inscribe{\sqsubseteq}{#1}}
\newcommand{\psel}[1]{\inscribe{\mid}{#1}}
\newcommand{\pmurec}[1]{\mu{#1}\ldotp\ }

\newcommand{\tr}[1]{\seq{#1}}

% \newcommand{\palpha}[1]{{\mbox{\texttt{α}}{#1}}}
% \newcommand{\tproc}[1]{{\mbox{\texttt{ι}}{#1}}}

\newcommand{\palpha}[1]{\text{\textgreek{α}}#1}
\newcommand{\tproc}[1]{\text{\textgreek{ι}}#1}

\newcommand{\tcat}[1]{\sqcat{#1}}

\newcommand{\tord}[1]{\inscribe{\le}{#1}}

\newcommand{\PC}[1]{\tuple{{\palpha{#1}} {\tproc{#1}}}}

\newcommand\sqcat[1]{%
  \begingroup%
  \def\inset{\cdot}%
  \def\mapitem{\identity}%
  \ensuremath{\insetmap{#1}}%
  \endgroup}

% \newcommand\seq[1]{\tuple{#1}}

% \newcommand\term[1]{\fn{TERM {\unite{#1}}}}

% \newcommand\term[1]{\fn{TERM #1}}
% \newcommand\tval[1]{\fn{val #1}}

\makeatletter

\def\extract@code#1#2\relax{#1}

% \newcommand\code[1]{\relax%
%   \expandafter\ifx\extract@code#1\relax\identity%
%     {#1}%
%   \else%
%     \mbox{\texttt{#1}}%
%   \fi\relax}

\newcommand\code[1]{\ensuremath{\codeitem{#1}}}

\makeatother

\newcommand\bbind[2]{\ensuremath{{#1}/{#2}}}
\newcommand\bind[2]{\ensuremath{[\bbind{#1}{#2}]}}
\newcommand\bindc[2]{\tuple{\bind{\code{#1}}{\code{#2}}}}

% \newcommand\codespace[0]{\mbox{\texttt{ }}}

\newcommand{\bred}[1]{\inscribe{\mathrel{{\to}_\beta}}{#1}}
\newcommand{\aeq}[1]{\inscribe{\mathrel{{=}_\alpha}}{#1}}

% \newcommand{\tuple}[1]{\seq{#1}}
% \newcommand\tlen[1]{\card{\dom{#1}}}
% \newcommand\tlen[1]{{#1_{\#}}}
\newcommand\tend[1]{{#1}_{\tlen{#1}-1}}

% \newcommand\tlen[1]{\capp{\mathrm{len} {#1}}}

%  \begingroup\def\inset{\times} \def\doword{} {\words#1}\endgroup}

\newcommand{\rel}[2]{\inscribe{\mathrel{#1}}{#2}}
\newcommand{\relation}[2]{\rel{#1}{#2}}

\newcommand{\card}[1]{\left|#1\right|}

\newcommand{\forevery}[1]{\forall{#1}\ldotp\ }
\newcommand{\thereis}[1]{\ensuremath{\exists{#1}\ldotp\ }}
% \newcommand{\lam}[1]{\text{\textgreek{λ}}#1\ldotp}

\newcommand{\coll}[2]{\left\{{{#1} \mid {#2}}\right\}}

\newcommand\nats{\mathbb{N}}

\newcommand{\bottom}[0]{{\bot}}
\newcommand{\dless}[0]{\sqsubseteq}
\newcommand{\dord}[1]{\inscribe{\dless}{#1}}
\newcommand{\dbot}[1]{{\bottom_{#1}}}

% \newcommand\all[1]{\inscribe{\wedge}{#1}}
% \newcommand\any[1]{\inscribe{\vee}{#1}}

\newcommand\all[1]{%
  \begingroup%
  \def\inset{\wedge}%
  \def\mapitem{\identity}%
  \ensuremath{\insetmap{#1}}%
  \endgroup}

\newcommand\any[1]{%
  \begingroup%
  \def\inset{\vee}%
  \def\mapitem{\identity}%
  \ensuremath{\insetmap{#1}}%
  \endgroup}

\newcommand\allop{\all{{} {}}}
\newcommand\anyop{\any{{} {}}}

% \newcommand{\join}[1]{\inscribe{\cap}{#1}}
% \newcommand{\unite}[1]{\inscribe{\cup}{#1}}

\newcommand\join[1]{%
  \begingroup%
  \def\inset{\cap}%
  \def\mapitem{\identity}%
  \ensuremath{\insetmap{#1}}%
  \endgroup}

\newcommand\unite[1]{%
  \begingroup%
  \def\inset{\cup}%
  \def\mapitem{\identity}%
  \ensuremath{\insetmap{#1}}%
  \endgroup}

\newcommand{\lub}[1]{\textstyle\bigsqcup{#1}}

\ifclassic
  \newcommand{\cpars}[1]{\left(#1\right)}
  \newcommand{\muset}[1]{\mu\left\{{#1}\right\}}
\else
  \newcommand{\cpars}[1]{#1}
  \newcommand{\muset}[1]{\mu{#1}}
\fi

\newcommand\fix[1]{{\mbox{fix}_{#1}\,}}
\newcommand\dom[1]{{\mbox{dom}\,{#1}}}


\newcommand\Nats{\nats}
% \newcommand\Nless[2]{{\all{{{#2} \in \Nats} {{#1} \le {#2}}}}}
% \newcommand\Nless[1]{{\nats_{{} < {#1}}}}

\newcommand\cpnmode[1]{\fn{mode #1}}
\newcommand\cpntype[1]{\fn{Type #1}}
\newcommand\cpnready[1]{\fn{ready #1}}

\newcommand\holesign{ 
  \mathchoice
    {\mbox{§}}
    {\mbox{§}}
    {\mbox{\scriptsize §}}
    {\mbox{\tiny §}}}

\newcommand\hole[1]{\holesign{#1}}

\newcommand\tlanext[1]{{{#1}'}}

\newcommand\tvar[1]{\fn{VAR #1}}
\newcommand\tlact[1]{\mathcal{#1}}
\newcommand\tlast[2]{\left[#1\right]_{#2}}
\newcommand\tlalways[1]{\square{#1}}

\newcommand\St{\mathbf{St}}

\newcommand\bydef{\overset{\mbox{\tiny def}}{=}}

\newcommand\iseq[3]{\tuple{#1}_{{#2}\in {#3}}}
\newcommand\nseq[2]{\iseq{#1}{#2}{\nats}}
\newcommand\bseq[4]{\iseq{\bind{#1_{#3}}{#2_{#3}}}{#3}{\Nless{#4}}}

\newcommand\cons[1]{\nomapinset{\cdot}{#1}}

\newcommand\R{\mathbf{R}}

\newcommand\tnothing{}
\newcommand\tsubst[1]{\left[#1\right]}
\newcommand\tval[1]{\splitinscribe\identity\identity\tsubst\tnothing#1\relax}

\newcommand\tlaval[1]{\splitinscribe\tlact\identity\tsubst\tnothing#1\relax}

\newcommand\upto[1]{\downarrow{#1}}
% \newcommand\uptoex[1]{\upto{({#1}-1)}}
% \newcommand\uptoex[1]{\text{\v{$\downarrow$}}{#1}}
\newcommand\uptoex[1]{{\hat{{\downarrow}}}{#1}}
% \newcommand\Nless[1]{\set{{\cdot}_{\Nats} < {#1}}}
\newcommand\Nless[1]{\uptoex{#1}}

\newcommand\stx[1]{\ensuremath{\text{#1}}}
\newcommand\bstx[1]{\ensuremath{{\text{#1}}_{\text{B}}}}

\newcommand\tcode[1]{\lstinline`#1`}
\newcommand\mcode[1]{\mbox{\tcode{#1}}}
\newcommand\mcsp{\mbox{\texttt{ }}}

\newcommand\codepair[2]{\(\frac{\mathrm{#1}}{\mathrm{#2}}\)}
% \newcommand\codepair[2]{\({\mathrm{#1}}_{\mathrm{#2}}\)}
% \newcommand\codepair[2]{{#1}/{#2}}

\newcommand\mpi[1]{\codepair{MPI}{#1}}
\newcommand\ride[1]{\codepair{RiDE}{#1}}
\newcommand\hadoop[1]{\codepair{Hadoop}{#1}}

\newcommand\uid[1]{\fn{{\text{\texttt{uid}}} {#1}}}
\newcommand\UID[1]{\fn{{\text{\texttt{UID}}} {#1}}}
\newcommand\node[1]{\fn{{\text{\texttt{node}}} {#1}}}

\newcommand\eval[1]{\llbracket {#1} \rrbracket}

\newenvironment{centercode}%
  {\begin{center}\begin{tabular}{c}}%
  {\end{tabular}\end{center}}

\newcommand\wa[1]{\ensuremath{{#1}_w}}

\newcommand\Input[1]{\fn{I {#1}}}
\newcommand\Wrte[1]{\fn{W {#1}}}
\newcommand\Img[1]{\text{Im}\,{#1}}

\providecommand\set[1]{\ensuremath{\{#1\}}}

\newcommand\In[1]{#1}
\newcommand\Wr[1]{#1}

% \newcommand\px[1]{\grk{p}-#1}


% \newcommand\im[1]{\capp{\mathrm{im} {#1}}}

\cfnscheme{im}{\mathrm{im}}
% \fnscheme{im}{\mathrm{im}}

% \newcommand\mr[1]{\ensuremath{\mathbf{R_{#1}}}}

\newcommand\mr[1]{\ensuremath{\mathtt{R}_\mathtt{#1}}}
