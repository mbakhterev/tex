% ДВИЖОК ПОДСТАНОВКИ. Этот набор макросов нужен для вставки различных
% конструкций между и вокруг элементами списка атомов (кажется, в терминологии
% TeX они называются токенами), разделённых пробелами. Например:
%
%   \simpleinset{\times}{\identity}{a bc de g}
%
% должно раскрываться в: \ensuremath{a \times bc \times de \times g}

\makeatletter

\def\im@end{\relax}
\def\insetmap#1{\im@loop#1 \im@end\relax}

% Здесь \begingroup и \endgroup использованы для ограничения области видимости
% определения \im@next

\def\im@loop#1 #2\relax{%
  \begingroup
  \ifx \im@end#2
    \def\im@next{\endgroup\mapitem{#1}\relax}
  \else%
    \def\im@next{\endgroup\mapitem{#1}\relax\expandafter\inset\im@loop#2\relax}%
  \fi
  \im@next
}

\def\headsplitmap#1{\hs@split#1 \im@end\relax}

\def\hs@split#1 #2\relax{%
  \ifx \im@end#2%
    \maphead{#1}{\pretail\posttail}%
  \else
    \maphead{#1}{\pretail\insetmap{#2}\posttail}}

\def\get@first#1#2\relax{#1}

\makeatother

\newcommand\simpleinset[3]{%
  \begingroup%
  \def\inset{#1}%
  \def\mapitem{#2}%
  \ensuremath{\insetmap{#3}}%
  \endgroup}

\newcommand\nomapinset[2]{\simpleinset{#1}{\identity}{#2}}

% ДВИЖОК ПОДСТАНОВКИ. КОНЕЦ
