% Домены

\makeatletter

\newcommand\domitem[1]{%
  \ifx\relax#1\relax
    {}
  \else\expandafter\ifx\get@first#1\relax\esc
    {#1}
  \else\expandafter\ifx\get@first#1\relax\Dom
    {#1}
  \else\expandafter\ifx\get@first#1\relax\DNm
    {#1}
  \else\expandafter\ifx\get@first#1\relax\KStar
    {#1}
  \else\expandafter\ifx\get@first#1\relax\KPlus
    {#1}
  \else\expandafter\ifx\get@first#1\relax\DProd
    {#1}
  \else\expandafter\ifx\get@first#1\relax\DSum
    {#1}
  \else\expandafter\ifx\get@first#1\relax\DSet
    {#1}
  \else
    \mathrm{#1}
  \fi\fi\fi\fi\fi\fi\fi\fi\fi}

\makeatother

% Стиль имён доменов

\newcommand\DNm[1]{\ensuremath{\domitem{#1}}}

\newcommand\Dom[1]{\ensuremath{{#1}_\bot}}

\newcommand\DBot[1]{\ensuremath{\bot_{\DNm{#1}}}}

\newcommand\DABot[1]{\ensuremath{\bot_{\atom{#1}}}}

% Скобки для доменных выражений

\newcommand\dset[1]{\Dom{\set{#1}}}
\newcommand\DSet[1]{\Dom{\Bras{#1}}}

\newcommand\dpars[1]{\Dom{\pars{#1}}}
\newcommand\DPars[1]{\Dom{\Pars{#1}}}

% Звёздочка Клини для переменных и доменов

\newcommand\kstar[1]{\ensuremath{{#1}^*}}
\newcommand\kplus[1]{\ensuremath{{#1}^+}}
\newcommand\KStar[1]{\kstar{\Dom{#1}}}
\newcommand\KPlus[1]{\kplus{\Dom{#1}}}

\dopscheme{DProd}{\times}{\prod}{\DNm}{\identity}{\identity}
\dopscheme{DSum}{+}{\sum}{\DNm}{\dpars}{\DPars}

% Произведения доменов

% \opscheme{prod}{\times}{\prod}

\newcommand\prj[2]{\ensuremath{\pi^{#1}_{\enumx{#2}}}}

% Суммы доменов

\dopscheme{oneof}{\sum}{+}{\identity}{\pars}{\Pars}

\newcommand\inj[2]{\ensuremath{\epsilon^{#1}_{\enumx{#2}}}}
