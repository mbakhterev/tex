% 3. Схемы перечислений для множеств и последовательностей. Используются
% примерно так:
%
%   \denumscheme{easet}{,}{\atom}{\set}
%
% 3-й параметра -- поэлементный декоратор; 4-й -- декоратор всего выражения.
%
% Не понятно, как здесь придать смысл i-комане, поэтому её в схеме нет.
% Остальное похоже на схему для операторов.

% Расстановка индексов 
\newcommand\Enum[4]{\ensuremath{{#1}_{{#2}={#3}}^{#4}}}
\newcommand\mEnum[4]{\ensuremath{{#1{#2}}_{{#2}={#3}}^{#4}}}

% #1 -- основа для команд
% #2 -- разделитель
\newcommand\enumscheme[2]{%
  \expandafter\newcommand\csname#1x\endcsname[1]{%
    \nomapinset{#2}{##1}}

  \expandafter\newcommand\csname#1r\endcsname[2]{%
    \nomapinset{#2}{{##1} {\ldots} {##2}}}
 
  \expandafter\newcommand\csname#1e\endcsname[3]{%
    \csname#1f\endcsname{##1}{##2}{1}{##3}}

  \expandafter\newcommand\csname#1f\endcsname[4]{%
    \Enum{##1}{##2}{##3}{##4}}

  \expandafter\newcommand\csname#1mx\endcsname[2]{%
    \simpleinset{#2}{##1}{##2}}

  \expandafter\newcommand\csname#1mr\endcsname[3]{%
    \nomapinset{#2}{{##1{##2}} {\ldots} {##1{##3}}}}

  \expandafter\newcommand\csname#1me\endcsname[3]{%
    \csname#1mf\endcsname{##1}{##2}{1}{##3}}

  \expandafter\newcommand\csname#1mf\endcsname[4]{%
    \mEnum{##1}{##2}{##3}{##4}}
}

% #1 -- основа для имён команд
% #2 -- разделитель
% #3 -- поэлементная декорация
% #4 -- декорация для всего выражения
\newcommand\denumscheme[4]{%
  \expandafter\newcommand\csname#1x\endcsname[1]{%
    #4{\simpleinset{#2}{#3}{##1}}}

  \expandafter\newcommand\csname#1r\endcsname[2]{%
    #4{\nomapinset{#2}{{#3{##1}} {\ldots} {#3{##2}}}}}
 
  \expandafter\newcommand\csname#1e\endcsname[3]{%
    \csname#1f\endcsname{##1}{##2}{1}{##3}}

  \expandafter\newcommand\csname#1f\endcsname[4]{%
    #4{\Enum{#3{##1}}{##2}{##3}{##4}}}

  \expandafter\newcommand\csname#1mx\endcsname[2]{%
    \begingroup%
    \newcommand\mdeco[1]{##1{#3}{####1}}%
    #4{\simpleinset{#2}{\mdeco}{##2}}%
    \endgroup}

  \expandafter\newcommand\csname#1mr\endcsname[3]{%
    #4{\nomapinset{#2}{{##1{#3}{##2}} {\ldots} {##1{#3}{##3}}}}}

  \expandafter\newcommand\csname#1me\endcsname[3]{%
    \csname#1mf\endcsname{##1}{##2}{1}{##3}}

  \expandafter\newcommand\csname#1mf\endcsname[4]{%
    \begingroup%
    \newcommand\mdeco[1]{##1{#3}{####1}}
    #4{\mEnum{\mdeco}{##2}{##3}{##4}}%
    \endgroup}
}

% Последовательности и множества

\enumscheme{enum}{,}

\opscheme{cup}{\cup}{\bigcup}
\opscheme{cap}{\cap}{\bigcap}
\opscheme{prod}{\times}{\prod}

\newcommand\tuple[1]{\ensuremath{\enumx{#1}}}
% \newcommand\Tuple[1]{\ensuremath{\left[\enumx{#1}\right]}}
% 
% \newcommand\seq[1]{\brkt{\enumx{#1}}}

\denumscheme{seq}{\ }{\identity}{\Brkt}
% \denumscheme{cseq}{,}{\identity}{\brkt}{\Brkt}

\newcommand\seq[1]{\ensuremath{\overline{#1}}}
