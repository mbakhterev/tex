% S-выражения. Команды и схемы. Схемы устроены следующим образом:
% \sexpscheme{keyword}. После чего строится набор команд со стандартными
% суффиксами i, [m]x, [m]e, [m]f. Эти команды рисуют s-выражение, в котором
% элементы записаны математически. Например,
%
%   \sexpscheme{run}\runx{a b} → \spars{\atom{run}\codespace a\codespace b}
%
% где, codespace и spars машинописные пробел и скобки. Кроме этой основной
% формы, определяется команда с суффиксом cx, в котором все элементы будут
% декорированы codeitem, и команды с суффиксами [m]bx, [m]be, [m]bf, в
% котором s-выражения не взяты в скобки spars. Это может быть полезно, как
% показывает документация Guile.
%
% В сложных декорированных схемах dsexpscheme пока не видно никакого смысла,
% потому что в s-выражениях и без того много оформления.
%
% Кроме схем нужны соответствующие конструкции для формирования произвольных
% s-выражений. Суффикс x в наборе этих команд не упомянут.

\newcommand\sexpscheme[1]{%
  \expandafter\newcommand\csname#1bxc\endcsname[1]
    {\simpleinset{\codespace}{\codeitem}{{#1} ##1}}

  \expandafter\newcommand\csname#1xc\endcsname[1]
    {\spars{\csname#1bxc\endcsname{##1}}}


  \expandafter\newcommand\csname#1bx\endcsname[1]
    {\nomapinset{\codespace}{\codeitem{#1} ##1}}

  \expandafter\newcommand\csname#1mbx\endcsname[2]
    {\ensuremath{\codeitem{#1}\codepace{\simpleinset{\codespace}{##1}{##2}}}}

  \expandafter\newcommand\csname#1x\endcsname[1]
    {\spars{\csname#1bx\endcsname{##1}}}

  \expandafter\newcommand\csname#1mx\endcsname[2]
    {\spars{\csname#1mbx\endcsname{##1}{##2}}}


  \expandafter\newcommand\csname#1br\endcsname[2]
    {\nomapinset{\codespace}{\codeitem{#1} {##1} {\ldots} {##2}}}

  \expandafter\newcommand\csname#1mbr\endcsname[3]
    {\csname#1br\endcsname{##1{##2}}{##1{##3}}}

  \expandafter\newcommand\csname#1r\endcsname[2]
    {\spars{\csname#1br\endcsname{##1}{##2}}}

  \expandafter\newcommand\csname#1mr\endcsname[3]
    {\spars{\csname#1mbr\endcsname{##1}{##2}{##3}}}


  \expandafter\newcommand\csname#1be\endcsname[3]
    {\csname#1bf\endcsname{##1}{##2}{1}{##3}}

  \expandafter\newcommand\csname#1mbe\endcsname[3]
    {\csname#1mbf\endcsname{##1}{##2}{1}{##3}}

  \expandafter\newcommand\csname#1e\endcsname[3]
    {\csname#1f\endcsname{##1}{##2}{1}{##3}}

  \expandafter\newcommand\csname#1me\endcsname[3]
    {\csname#1mf\endcsname{##1}{##2}{1}{##3}}


  \expandafter\newcommand\csname#1bf\endcsname[4]
    {\codeitem{#1}\codespace\sexpbf{##1}{##2}{##3}{##4}}

  \expandafter\newcommand\csname#1mbf\endcsname[4]
    {\codeitem{#1}\codespcae\sexpmbf{##1}{##2}{##3}{##4}}

  \expandafter\newcommand\csname#1f\endcsname[4]
    {\spars{\csname#1bf\endcsname{##1}{##2}{##3}{##4}}}

  \expandafter\newcommand\csname#1mf\endcsname[4]
    {\spars{\csname#1mbf\endcsname{##1}{##2}{##3}{##4}}}
}

\def\codespace{\text{\texttt{ }}}

\newcommand\sexpbc[1]{\simpleinset{\codespace}{\codeitem}{#1}}
\newcommand\sexpc[1]{\spars{\sexpbc{#1}}}

\newcommand\sexpb[1]{\nomapinset{\codespace}{#1}}
\newcommand\sexpmb[2]{\simpleinset{\codespace}{#1}{#2}}
\newcommand\sexp[1]{\spars{\sexpb{#1}}}
\newcommand\sexpm[2]{\spars{\sexpmb{#1}{#2}}}

\newcommand\sexpbr[2]{\nomapinset{\codespace}{{#1} {\ldots} {#2}}}
\newcommand\sexpmbr[3]{\nomapinset{\codespace}{{#1{#2}} {\ldots} {#1{#3}}}}
\newcommand\sexpr[2]{\spars{\sexpbr{#1}{#2}}}
\newcommand\sexpmr[3]{\spars{\sexpmbr{#1}{#2}{#3}}}

\newcommand\sexpbe[3]{\sexpbf{#1}{#2}{1}{#3}}
\newcommand\sexpmbe[4]{\sexpmbe{#1}{#2}{#3}{#4}}
\newcommand\sexpe[3]{\sexpf{#1}{#2}{1}{#3}}
\newcommand\sexpme[4]{\sexpmf{#1}{#2}{#3}{#4}}

\newcommand\sexpbf[4]{\ensuremath{{#1}_{{#2} = {#3}}^{#4}}}
\newcommand\sexpmbf[4]{\ensuremath{{#1{#2}}_{{#2} = {#3}}^{#4}}}
\newcommand\sexpf[4]{\spars{\sexpbf{#1}{#2}{#3}{#4}}}
\newcommand\sexpmf[4]{\spars{\sexpmbf{#1}{#2}{#3}{#4}}}

\makeatletter

\newcommand\codeitem[1]{%
  \ifx\relax#1\relax
    \text{\texttt{ }}
  \else\expandafter\ifx\get@first#1\relax\esc
    {#1}
  \else\expandafter\ifx\get@first#1\relax\sexp
    {#1}
  \else\expandafter\ifx\get@first#1\relax\satom
    {#1}
  \else\expandafter\ifx\get@first#1\relax\atom
    {#1}
  \else
    \text{\texttt{#1}}
  \fi\fi\fi\fi\fi}

\makeatother

\newcommand\baresexp[1]{%
  \begingroup%
  \def\inset{\text{\texttt{\ }}}%
  \def\mapitem{\codeitem}%
  \ensuremath{\insetmap{#1}}%
  \endgroup}

% \newcommand\satom[1]{\ensuremath{\text{\texttt{#1}}}}
% \newcommand\sexp[1]{\ensuremath{\text{\texttt{(}}\baresexp{#1}\text{\texttt{)}}}}

% \newcommand\sexp[1]{\spars{\baresexp{#1}}}

\newcommand\atom[1]{\ensuremath{\text{\texttt{#1}}}}

\newcommand\lexp[2]{\sexp{\grk{l} #1 #2}}

\newcommand\fn[1]{\app{#1}}

\newcommand\clam[0]{\ensuremath{\text{\texttt{λ}}}}
\newcommand\fapp[1]{\sexp{#1}}
\newcommand\fabs[2]{\sexp{\clam{} {#1} #2}}
