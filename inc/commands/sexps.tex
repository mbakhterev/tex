% S-выражения. Команды и схемы. Схемы устроены следующим образом:
% \sexpscheme{keyword}. После чего строится набор команд со стандартными
% суффиксами i, [m]x, [m]e, [m]f. Эти команды рисуют s-выражение, в котором
% элементы записаны математически. Например,
%
%   \sexpscheme{run}\runx{a b} → \spars{\atom{run}\codespace a\codespace b}
%
% где, codespace и spars машинописные пробел и скобки. Кроме этой основной
% формы, определяется команда с суффиксом cx, в котором все элементы будут
% декорированы codeitem, и команды с суффиксами [m]bx, [m]be, [m]bf, в
% котором s-выражения не взяты в скобки spars. Это может быть полезно, как
% показывает документация Guile.
%
% В сложных декорированных схемах dsexpscheme пока не видно никакого смысла,
% потому что в s-выражениях и без того много оформления.

\makeatletter

\newcommand\codeitem[1]{%
  \ifx\relax#1\relax
    \text{\texttt{ }}
  \else\expandafter\ifx\get@first#1\relax\esc
    {#1}
  \else\expandafter\ifx\get@first#1\relax\sexp
    {#1}
  \else\expandafter\ifx\get@first#1\relax\satom
    {#1}
  \else\expandafter\ifx\get@first#1\relax\atom
    {#1}
  \else
    \text{\texttt{#1}}
  \fi\fi\fi\fi\fi}

\makeatother

\newcommand\baresexp[1]{%
  \begingroup%
  \def\inset{\text{\texttt{\ }}}%
  \def\mapitem{\codeitem}%
  \ensuremath{\insetmap{#1}}%
  \endgroup}

% \newcommand\satom[1]{\ensuremath{\text{\texttt{#1}}}}
% \newcommand\sexp[1]{\ensuremath{\text{\texttt{(}}\baresexp{#1}\text{\texttt{)}}}}

\newcommand\sexp[1]{\spars{\baresexp{#1}}}
\newcommand\atom[1]{\ensuremath{\text{\texttt{#1}}}}
\newcommand\lexp[2]{\sexp{\grk{l} #1 #2}}

\newcommand\fn[1]{\app{#1}}

\newcommand\clam[0]{\ensuremath{\text{\texttt{λ}}}}
\newcommand\fapp[1]{\sexp{#1}}
\newcommand\fabs[2]{\sexp{\clam{} {#1} #2}}
